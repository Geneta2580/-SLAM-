\ExplSyntaxOn
\cs_generate_variant:Nn \fp_set:Nn { NV }
\cs_generate_variant:Nn \fp_gset:Nn { NV }
\ExplSyntaxOff

\documentclass[12pt, a4paper, oneside]{ctexart}
\usepackage{amsmath, amsthm, amssymb, appendix, bm, graphicx, hyperref, mathrsfs, geometry, indentfirst, graphicx, diagbox}
\geometry{a4paper,left=2cm,right=2cm,top=3cm,bottom=3cm}  % 修改页边距

\usepackage{titlesec} %自定义多级标题格式的宏包
\usepackage{hyperref}
\hypersetup{hidelinks,
	colorlinks=true,
	allcolors=black,
	pdfstartview=Fit,
	breaklinks=true}
\titleformat{\section}[block]{\Large\bfseries}{\zhnum{section}.}{1em}{}[] 
% \zhum中文编号 \arabic 数字编号

\title{\vspace{-4cm}\bfseries{CH5 HOMEWORK}}
\author{\large 庞骏翔 \quad ZY2417209}  % \quad 空格一个字符 \large 小四 \bfseries 黑体
\date{}
\linespread{1.5}  % 修改行距
\setlength{\parindent}{2em} % 段首缩进两字符
\begin{document}
	
	\pagestyle{plain}
	\maketitle
	\section{2 ORB特征点}
	
	1、ORB提取
	
	2、ORB描述
	注意这里p、q取的是相对坐标,应对$p^{'}$、$q^{'}$做边界检测
	
	3、暴力匹配
	
	见GenetaSLAM Project homework ch5 computeORB
	
	\section{3 从E恢复R、t}
	
	见GenetaSLAM Project homework ch5 E2Rt
	
	\section{4 GN实现Bundle Adjustment}
	
	见GenetaSLAM Project homework ch5 GN-BA
	
	\section{5 ICP实现轨迹对齐}

	见GenetaSLAM Project homework ch5 ICP
	
\end{document}