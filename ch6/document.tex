\ExplSyntaxOn
\cs_generate_variant:Nn \fp_set:Nn { NV }
\cs_generate_variant:Nn \fp_gset:Nn { NV }
\ExplSyntaxOff

\documentclass[12pt, a4paper, oneside]{ctexart}
\usepackage{amsmath, amsthm, amssymb, appendix, bm, graphicx, hyperref, mathrsfs, geometry, indentfirst, graphicx, diagbox}
\geometry{a4paper,left=2cm,right=2cm,top=3cm,bottom=3cm}  % 修改页边距

\usepackage{titlesec} %自定义多级标题格式的宏包
\usepackage{hyperref}
\hypersetup{hidelinks,
	colorlinks=true,
	allcolors=black,
	pdfstartview=Fit,
	breaklinks=true}
\titleformat{\section}[block]{\Large\bfseries}{\zhnum{section}.}{1em}{}[] 
% \zhum中文编号 \arabic 数字编号

\title{\vspace{-4cm}\bfseries{CH6 HOMEWORK}}
\author{\large 庞骏翔 \quad ZY2417209}  % \quad 空格一个字符 \large 小四 \bfseries 黑体
\date{}
\linespread{1.5}  % 修改行距
\setlength{\parindent}{2em} % 段首缩进两字符
\begin{document}
	
	\pagestyle{plain}
	\maketitle
	\section{2 LK光流}
	
	1、光流文献综述
	(1)光流法分为additive和compositional,forwards和inverse
	(2)应该是warp,疑似是为了更好近似图像物理层面的扭曲
	(3)forward没啥好多说的,就是一般优化的流程,给出误差目标函数,根据变量对误差进行求导,然后找到使误差最小的变量变化量,
	inverse的动机在于,forward每次计算导数时,都会需要重新计算一个H(海塞)矩阵,这样做的计算量相当大,能否找到一种等价方式,使计算导数并更新时,H阵近似于一个常数阵(当然由于求导更新中其它差值变量得存在,求导求差的过程不会被这个常值阵给吃掉),inverse这里交换了模板和图像的角色(交换前一时刻和后一时刻的顺序),导致求导等于0之后H的封闭解形式和自变量无关,可以近似成一个常值阵的形式,非常妙,但是对应的,对仿射变换的W矩阵有一定的要求
	
	2、forward-addtive G-N光流实现
	(1)前一时刻的像素灰度值和后一时刻窗口内(以前一时刻像素点为中心)像素灰度值之差的平方之和
	(2)和雅可比矩阵有关,直接对自变量求导,保留p即可,矩阵中肯定是包含x、y方向上的像素梯度的内容的
	
	3、反向法
	
	4、推广至金字塔
	(1)是指先从金字塔的最上层,也就是尺寸最小的图片进行光流的计算,对从底层提取的特征点坐标进行相应放缩,然后根据放缩后的特征点窗口进行灰度差值计算,进行优化,得出在第二张图中的光流以及特征点,然后像这样逐层下降即为coarse-to-fine的过程
	(2)特征点法的金字塔是为了解决特征点尺度不变性的问题,所以匹配不同层上的图像,实现尺度不变性
	而光流法的金字塔是为了解决若相机运动较快,则两张图像差异明显,缩放后则差异减小,能够提高算法效果
	
	5、讨论
	(1)光照发生变化的时候不合理?是不是得整体看光照是否发生变化,若发生变化,是不是可以加上一个针对光照的误差偏移量
	(2)必然会有明显差异,窗口大小还是非常重要的,感觉图像块越大捕捉的光流越容易不准,但是对于相机运动速度快的可能适应性更好
	(3)金字塔层数越多应该结果也是越准的,过渡更平滑,缩放倍率越小越平滑,但缩放倍率增大可以减少计算开销?
	
	见GenetaSLAM Project homework ch6 optical flow
	
	\section{3 直接法}
	
	1、单层直接法
	(1)误差项是模板图像(前一时刻图像)和投影图像(当前图像)中像素灰度值之差,损失为二范数之和
	(2)误差对于自变量的雅可比纬度误差是一个一维(1x1)的数据,自变量是一个是一个1x2的矩阵和2x6的矩阵的乘积,所以应该是一个1x6的矩阵
	(3)在本代码中,窗口取的是4x4的patch,最大可以取到缩放之后顶层的金字塔,大窗口意味着计算量的急剧增加(需要遍历窗口图像内的每一个点),可以取单个点进行计算,但是无法利用区域内其它像素点的灰度值,这使算法极易受到光照的影响,具体来说,当两张要匹配的图像光照发生变化时,由于单点的灰度差发生显著变化,因此灰度不变假设失效,正确的匹配点不会被识别到,优化的相机位姿也会因此不正确,而窗口内的灰度全部变化后,灰度不变假设变为取一个窗口内变化的最小值,因此仍然能保持算法的正确性
	2、多层直接法
	
	\section{3 *延伸讨论}
	(1)可以,同样能够减小计算量,提高计算效率
	(2)每个点的优化计算可以并行执行,充分利用多线程并行的性质,加速优化的进程
	(3)参考图像(前一张图像)的一个patch和后一张图像对应的投影patch灰度值近似不变(最小化)
	(4)直接法更多关注的是区域像素的亮度信息,注重的是图像整体的一致性,特征可以较为稠密,而特征点法注重的是图像的几何特征,是一种稀疏特征的方法
	(5)直接法不太受光照等结构因素变化的影响,但是由于其是一种稠密特征的方法,需要的计算开销比较大,而特征点法计算开销比较小,但是受环境影响较大
	
	见GenetaSLAM Project homework ch6 direct method
	
	\section{4 *使用光流计算视差}
	
	见GenetaSLAM Project homework ch6 ICP
	
\end{document}