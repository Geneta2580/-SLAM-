\ExplSyntaxOn
\cs_generate_variant:Nn \fp_set:Nn { NV }
\cs_generate_variant:Nn \fp_gset:Nn { NV }
\ExplSyntaxOff

\documentclass[12pt, a4paper, oneside]{ctexart}
\usepackage{amsmath, amsthm, amssymb, appendix, bm, graphicx, hyperref, mathrsfs, geometry, indentfirst, graphicx, diagbox}
\geometry{a4paper,left=2cm,right=2cm,top=3cm,bottom=3cm}  % 修改页边距

\usepackage{titlesec} %自定义多级标题格式的宏包
\usepackage{hyperref}
\hypersetup{hidelinks,
	colorlinks=true,
	allcolors=black,
	pdfstartview=Fit,
	breaklinks=true}
\titleformat{\section}[block]{\Large\bfseries}{\zhnum{section}.}{1em}{}[] 
% \zhum中文编号 \arabic 数字编号

\title{\vspace{-4cm}\bfseries{关于开题的一些想法}}
\author{\large 庞骏翔 \quad ZY2417209}  % \quad 空格一个字符 \large 小四 \bfseries 黑体
\date{}
\linespread{1.5}  % 修改行距
\setlength{\parindent}{2em} % 段首缩进两字符
\begin{document}
	
	\pagestyle{plain}
	\maketitle
	\section{2 熟悉Eigen矩阵运算}
	
	3、任意复(实)矩阵A分解为正交(酉)矩阵Q和上三角阵R,解$Ax=b$,分解A,则$QRx=b$,解得$x=R^{-1}Q^{T}b$,Householder方法稳定且被广泛使用
	(还有经典的Gram-Schmidt正交化方法)
	
	4、对于对称正定矩阵A分解为$A=LL^{T}$,L的对角线元素均为正数,L是下三角矩阵,解$Ax=b$,分解A,解$Ly=b$,解$L^{T}x=y$
	
	5、见GenetaSLAM Project homework ch2 涉及到Eigen库的使用,求解方程
	
	\section{3 几何运算练习}
	
	见GenetaSLAM Project homework ch2 涉及到Eigen库,基本的旋转以及平移的坐标变换
	
	\section{4 旋转的表达}
	
	1、略,旋转矩阵DCM必为正交阵,可以从DCM定义出发(那个坐标乘单位向量的坐标变换形式,转置求变换系下的坐标导出)
	
	2、虚部3d,实部1d
	
	3、
	1. 首先回顾四元数的基本形式:

	- 一个四元数\(q = \eta+\epsilon_x i+\epsilon_y j+\epsilon_z k\),可以写成\(q=\left[\begin{array}{c}\eta\\\epsilon\end{array}\right]\),其中\(\eta\in R\),\(\epsilon = [\epsilon_x,\epsilon_y,\epsilon_z]^T\in R^3\)。对于单位四元数,有\(\eta^{2}+\epsilon^{T}\epsilon = 1\)。
	
	- 设\(q_1=\eta_1+\epsilon_1\),\(q_2=\eta_2+\epsilon_2\),根据四元数乘法规则:

	- \(q_1\cdot q_2=(\eta_1+\epsilon_1)(\eta_2+\epsilon_2)=\eta_1\eta_2-\epsilon_1^{T}\epsilon_2+\eta_1\epsilon_2+\eta_2\epsilon_1+\epsilon_1\times\epsilon_2\)。

	2. 然后计算\(q_1^{+}q_2\):

	- 已知\(q_1^{+}=\left[\begin{array}{cc}\eta_1 - \epsilon_1^{\times}&\epsilon_1\\-\epsilon_1^{T}&\eta_1\end{array}\right]\),\(q_2=\left[\begin{array}{c}\eta_2\\\epsilon_2\end{array}\right]\)。

	- 根据矩阵乘法:

	- \(q_1^{+}q_2=\left[\begin{array}{cc}\eta_1 - \epsilon_1^{\times}&\epsilon_1\\-\epsilon_1^{T}&\eta_1\end{array}\right]\left[\begin{array}{c}\eta_2\\\epsilon_2\end{array}\right]=\left[\begin{array}{c}(\eta_1 - \epsilon_1^{\times})\eta_2+\epsilon_1\epsilon_2\\-\epsilon_1^{T}\eta_2+\eta_1\epsilon_2\end{array}\right]\)。

	- 其中\(\epsilon_1^{\times}\)是\(\epsilon_1\)对应的反对 - 称矩阵,\((\epsilon_1^{\times})_{ij}=-\epsilon_{1k}\)(\((i,j,k)\)是\((1,2,3)\)的循环排列)。

	- \((\eta_1 - \epsilon_1^{\times})\eta_2+\epsilon_1\epsilon_2=\eta_1\eta_2-\epsilon_1^{\times}\eta_2+\epsilon_1\epsilon_2=\eta_1\eta_2-\epsilon_1^{T}\epsilon_2+\eta_1\epsilon_2+\eta_2\epsilon_1+\epsilon_1\times\epsilon_2\)(利用反对称矩阵与向量乘法和叉乘的关系:\(\epsilon_1^{\times}\eta_2 =-\eta_2\epsilon_1-\epsilon_1\times\epsilon_2\)),\(-\epsilon_1^{T}\eta_2+\eta_1\epsilon_2\)也符合四元数乘法结果的向量部分形式,所以\(q_1\cdot q_2 = q_1^{+}q_2\)。
	
	3. 接着计算\(q_2^{\oplus}q_1\):
	
	- 已知\(q_2^{\oplus}=\left[\begin{array}{cc}\eta_2+\epsilon_2^{\times}&\epsilon_2\\-\epsilon_2^{T}&\eta_2\end{array}\right]\),\(q_1=\left[\begin{array}{c}\eta_1\\\epsilon_1\end{array}\right]\)。

	- 根据矩阵乘法:
	- \(q_2^{\oplus}q_1=\left[\begin{array}{cc}\eta_2+\epsilon_2^{\times}&\epsilon_2\\-\epsilon_2^{T}&\eta_2\end{array}\right]\left[\begin{array}{c}\eta_1\\\epsilon_1\end{array}\right]=\left[\begin{array}{c}(\eta_2+\epsilon_2^{\times})\eta_1+\epsilon_2\epsilon_1\\-\epsilon_2^{T}\eta_1+\eta_2\epsilon_1\end{array}\right]\)。

	- 利用反对称矩阵与向量乘法和叉乘的关系\(\epsilon_2^{\times}\eta_1=-\eta_1\epsilon_2 - \epsilon_2\times\epsilon_1\),可得\((\eta_2+\epsilon_2^{\times})\eta_1+\epsilon_2\epsilon_1=\eta_1\eta_2-\epsilon_1^{T}\epsilon_2+\eta_1\epsilon_2+\eta_2\epsilon_1+\epsilon_1\times\epsilon_2\),\(-\epsilon_2^{T}\eta_1+\eta_2\epsilon_1\) 也符合四元数乘法结果的向量部分形式,所以\(q_1\cdot q_2 = q_2^{\oplus}q_1\)。
	
	综上,通过利用四元数乘法规则以及反对称矩阵与向量乘法和叉乘的关系,证明了对任意单位四元数\(q_1,q_2\),\(q_1\cdot q_2 = q_1^{+}q_2\)和\(q_1\cdot q_2 = q_2^{\oplus}q_1\)成立。
	
	\section{5 罗德里格斯公式的证明}
	
	略

	\section{6 四元数运算性质的验证}
	
	用第4题的结论,结合四元数的实部虚部既能得到结论
	
\end{document}