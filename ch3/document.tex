\ExplSyntaxOn
\cs_generate_variant:Nn \fp_set:Nn { NV }
\cs_generate_variant:Nn \fp_gset:Nn { NV }
\ExplSyntaxOff

\documentclass[12pt, a4paper, oneside]{ctexart}
\usepackage{amsmath, amsthm, amssymb, appendix, bm, graphicx, hyperref, mathrsfs, geometry, indentfirst, graphicx, diagbox}
\geometry{a4paper,left=2cm,right=2cm,top=3cm,bottom=3cm}  % 修改页边距

\usepackage{titlesec} %自定义多级标题格式的宏包
\usepackage{hyperref}
\hypersetup{hidelinks,
	colorlinks=true,
	allcolors=black,
	pdfstartview=Fit,
	breaklinks=true}
\titleformat{\section}[block]{\Large\bfseries}{\zhnum{section}.}{1em}{}[] 
% \zhum中文编号 \arabic 数字编号

\title{\vspace{-4cm}\bfseries{CH3 HOMEWORK}}
\author{\large 庞骏翔 \quad ZY2417209}  % \quad 空格一个字符 \large 小四 \bfseries 黑体
\date{}
\linespread{1.5}  % 修改行距
\setlength{\parindent}{2em} % 段首缩进两字符
\begin{document}
	
	\pagestyle{plain}
	\maketitle
	\section{2 群的性质}
	
	1、$\mathbb{Z}$对于加法封闭,且都符合群的性质
	
	2、$\mathbb{N}$对于加法封闭,但其不符合群的性质中逆元的性质
	
	\section{3 验证向量叉乘的李代数性质}
	
	略
	
	封闭性:设三维向量,,按照叉乘的定义进行运算得到结果仍属于$\mathbb{R}^{3}$即可
	
	双线性:设向量$a$、$b$、$c$,将李括号替换为叉乘进行运算即可
	
	自反性:按照叉乘计算自己叉乘自己即可
	
	雅可比等价:设向量$a$、$b$、$c$,将李括号替换为叉乘进行运算即可
	
	\section{4 推导 SE(3) 的指数映射}
	
	利用反对称矩阵乘幂的性质,将高次幂转为低次幂,注意$\phi=\theta a$,$\theta$为常数,$a$为单位向量,利用$\phi^{\wedge}=\theta a^{\wedge}$,展开即可
	
	\section{5 伴随}
	
	设\(\boldsymbol{a}=(a_x,a_y,a_z)^T\in\mathbb{R}^3\),\(\boldsymbol{R}\in SO(3)\),\(\boldsymbol{a}^{\wedge}=\begin{bmatrix}0& - a_z&a_y\\a_z&0& - a_x\\-a_y&a_x&0\end{bmatrix}\)是\(\boldsymbol{a}\)对应的反对称矩阵。
	
	对于任意向量\(\boldsymbol{v}\in\mathbb{R}^3\),有:

	\begin{align}
		\boldsymbol{R}\boldsymbol{a}^{\wedge}\boldsymbol{R}^T\boldsymbol{v}&=\boldsymbol{R}(\boldsymbol{a}^{\wedge}(\boldsymbol{R}^T\boldsymbol{v}))\\
		&=\boldsymbol{R}(\boldsymbol{a}\times(\boldsymbol{R}^T\boldsymbol{v}))
	\end{align}
	
	根据向量叉乘的性质\(\boldsymbol{u}\times\boldsymbol{w} = -\boldsymbol{w}\times\boldsymbol{u}\)和旋转矩阵\(\boldsymbol{R}\)的性质(保持向量叉乘关系),\(\boldsymbol{R}(\boldsymbol{a}\times\boldsymbol{b})=(\boldsymbol{R}\boldsymbol{a})\times(\boldsymbol{R}\boldsymbol{b})\),则:
	\begin{align}
		\boldsymbol{R}(\boldsymbol{a}\times(\boldsymbol{R}^T\boldsymbol{v}))&= (\boldsymbol{R}\boldsymbol{a})\times(\boldsymbol{R}\boldsymbol{R}^T\boldsymbol{v})\\
		&=(\boldsymbol{R}\boldsymbol{a})\times\boldsymbol{v}\\
		&= (\boldsymbol{R}\boldsymbol{a})^{\wedge}\boldsymbol{v}
	\end{align}
	
	因为对于任意\(\boldsymbol{v}\in\mathbb{R}^3\)都成立,所以\(\boldsymbol{R}\boldsymbol{a}^{\wedge}\boldsymbol{R}^T = (\boldsymbol{R}\boldsymbol{a})^{\wedge}\)。
	
	\section{6 轨迹的描绘}
	
	见GenetaSLAM Project homework ch3
	
\end{document}